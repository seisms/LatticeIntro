% Aquí definí algunos comandos personalizados.
% Los más importantes son: \latt que escribe L(B) y
% \lattprime -> L(B').
% Otros son \ZZ, \RR, \QQ etc. que escriben los enteros, reales
% etc.
\documentclass{beamer}

\usepackage{subcaption, float}
\usepackage{graphicx} % Required for inserting images
\usepackage{svg}
\usepackage[spanish]{babel}

%%%%%%%%%% Font %%%%%%%%%%%
\usepackage[T1]{fontenc}
\usepackage{amssymb, amsfonts}
\usepackage{amsbsy}
%%%%%%%%%%%%%%%%%%%%%%%%%%%


%%%%%%%%%% Math %%%%%%%%%%%
% Useful tools like equation environments, alignment, etc.
\usepackage{amsthm, amsmath}

\theoremstyle{remark}
\newtheorem*{remark}{Observación}

\theoremstyle{plain}
\newtheorem{proposition}{Proposición}

% Para cambiar a los símbolos "tradicionales" de
% de los naturales reales etc. se tiene que cambiar
% \mathbf por \mathbb. Me gusta más dejarlos en negrita
% porque encuentro que es más legible en presentaciones,
% pero si quieren lo cambian.
\newcommand{\ZZ}{\mathbb{Z}}
\newcommand{\QQ}{\mathbb{Q}}
\newcommand{\RR}{\mathbb{R}}
\newcommand{\MM}{\mathbb{M}}
\newcommand{\NN}{\mathbb{N}}
\newcommand{\latt}{\mathcal{L}(\mathbf{B})}       % Para escribir menos
\newcommand{\lattprime}{\mathcal{L}(\mathbf{B'})} % Para escribir menos

%%%%%%%%%%%%%%%%%%%%%%%%%%%%


%%%% Beamer specific %%%%%%
\title{Introducción a Reticulados}
% Pongan sus apellidos maternos en ...
\author{Leandro Aballay Henríquez, Diego Cuevas Alarcón, Matías Olivares Morales, Álvaro Quezada Inostroza, Guillermo Pereira Bula}
\institute{Departamento de Matemática y Ciencia de la Computación\\}
\date{17 de julio de 2025}

\AtBeginSection[]
{
  \begin{frame}
    \frametitle{Contenidos}
    \tableofcontents[currentsection]
  \end{frame}
}

\uselanguage{Spanish}
\languagepath{Spanish}
\newcommand{\adv}[1]{ {\color{blue} #1} }


\usetheme{default}
\usecolortheme{default}
%%%%%%%%%%%%%%%%%%%%%%%%%%%


\begin{document}
\frame{\titlepage}

\begin{frame}
\frametitle{Contenidos}
\tableofcontents
\end{frame}

\section{Marco teórico}


% Preliminares acá? Definir la generalización de norma y conjunto
% abierto? Podría ayudar cuando hablemos de mínimos sucesivos.
\begin{frame}{Preliminares}
Denotamos un vector en algún dominio como $\mathbf x$, y los componentes de este vector $v_i$ para el $i$-ésimo componente. Asimismo, $\mathbf M$ denota una matriz.

A menos que sea especificado, la norma $||\cdot ||: \RR^n \to \RR$ denota la norma $l_2$ (o norma Euclidiana).

\end{frame}

% Por ahora estoy poniendo las cosas nomas. De ahí les podemos poner titulo
% a los frames o q se yo, refinar la presentacion etc.
\begin{frame}
\begin{definition}[Reticulado]
Sea $\RR^m$ el espacio euclidiano $m$-dimensional. Un reticulado en $\RR^m$ es el conjunto
\[
\mathcal L(\mathbf{b}_1, \dots, \mathbf{b}_n) = \left\{\sum_{i = 1}^n x_i \mathbf{b}_i : x_i  \in \ZZ\right\}
\]
De combinaciones lineales enteras de $n$ vectores linealmente independientes $\mathbf{b}_1, ..., \mathbf{b}_n$ en $\RR^m$ con $m \geq n$.
\end{definition}
Llamamos a los enteros $m$ y $n$ como {\it dimensión} y {\it rango} respectivamente. Si $m = n$, es decir, si el número de vectores base es igual a la dimensión del espacio, en2tonces diremos que $\latt$ es de rango completo, o dimensión completa.
\end{frame}

\begin{frame}
\only<1>{Los vectores $\mathbf{b}_1, ..., \mathbf{b}_n$ son la {\it} base del reticulado. Podemos representarlos como una matriz $\RR^{m\times n}$ donde los vectores base ``se ven'' como columnas

\[
\mathbf{B} = \begin{bmatrix}
    \mathbf{b}_1 & \dots & \mathbf{b}_n
\end{bmatrix}
\]

De esta forma compactamos la primera definición como $\mathcal{L}(\mathbf B) = \left\{\mathbf{Bx} : \mathbf x \in \ZZ^n\right\}$. 

Un reticulado tiene muchas bases distintas. Dos bases $\mathbf B$ y $\mathbf B'$ que generan el mismo reticulado se dicen {\it equivalentes}.
}
\only<2>{
\begin{figure}
    
    \centering
    
    \begin{subfigure}{0.4\textwidth}
        \includesvg[width=\linewidth]{figures/lattice_fig.svg}
        \caption{$\mathbf B = [\mathbf b_1\quad \mathbf b_2]$}
        \label{fig:simple_lattice}
    \end{subfigure}
    \hfill
    \begin{subfigure}{0.4\textwidth}
        \includesvg[width=\linewidth]{figures/lattice_bprime.svg}
        \caption{$\mathbf B' = [\mathbf b'_1\quad \mathbf b'_2]$}
        \label{fig:same_basis}
    \end{subfigure}
    \caption{Dos bases equivalentes.}
\end{figure}
}
\end{frame}

\begin{frame}{Amplitud}
La {\it amplitud} (o {\it span}) de la base de un reticulado se define como el conjunto de todas las combinaciones lineales {\it reales} de los vectores base:
\[
\text{span}(\mathbf B) = \left\{\mathbf{Bx} : \mathbf x \in \RR^n \right\}
\]

$\mathbf B$ es una base para $\text{span}(\mathbf{B})$ visto como un espacio vectorial. En particular, esto implica que $\text{rank}(\latt) = \text{dim}(\text{span}(\mathbf B))$. Asimismo, $\text{span}(\mathbf B) = \RR^m$ si su dimensión es $m$. De lo anterior podemos concluir que $\latt$ es de rango completo si y solo si $\text{span}(\mathbf B) = \RR^m$.

\begin{remark}
Cualquier conjunto de $n$ vectores L.I. $\mathbf B'$ del reticulado $\latt$ es una base para $\text{span}(\mathbf B)$, pero $\mathbf B'$ no es necesariamente equivalente a $\mathbf B$.
\end{remark}
\end{frame}

% La corta demostración
\begin{frame}
\only<1>{
\begin{theorem}
Sean $\mathbf B$ y $\mathbf B'$ dos bases $\RR^{m \times n}$. $\mathbf B$ y $\mathbf B'$ son equivalentes si y solo si existe una matriz unimodular $\mathbf U \in \ZZ^{n \times n}$ tal que $\mathbf B = \mathbf {B' U}$
\end{theorem}
}
\begin{proof}
 \only<1>{$(\impliedby)$ Tenemos que $\mathbf B = \mathbf{B'U}$ y $\det(\mathbf U) = \pm 1 \neq 0$, entonces $\mathbf U$ es invertible por alguna matriz $\mathbf U'$ tal que $\mathbf{UU'} = \mathbf I$. En particular, de esto podemos concluir que $\mathbf{B'} = \mathbf{BU'}$. Sigue inmediatemente que $\latt \subseteq \lattprime$ y $\lattprime \subseteq \latt$. Por lo tanto $\latt = \lattprime$ y ambas bases son equivalentes.}

 \only<2>{$(\implies)$ Si $\mathbf B$ y $\mathbf B'$ son equivalentes entonces generan el mismo reticulado. Es decir, existen $\mathbf M$ y $\mathbf M' \in \ZZ^{n\times n}$  tal que $\mathbf B = \mathbf{B'M'}$ y $\mathbf B' = \mathbf{BM}$.
 \[
 \mathbf B = \mathbf{BMM'} \implies \mathbf{B}(\mathbf{I - MM'}) = \mathbf 0
 \]
 Sabemos que $\mathbf B$ es linealmente independiente. Por lo tanto $(\mathbf{I- MM'}) = \mathbf 0$. Luego 
\[ 
 \det(\mathbf{MM'}) = \det(\mathbf M) \cdot \det(\mathbf M') = \det(\mathbf{I}) = 1
 \]

Como $\mathbf M$ y $\mathbf M'$ están compuestas de enteros, $\det(\mathbf M)$ y $\det(\mathbf{M'})$ deben ser enteros. Como la multiplicación de ambos determinantes debe ser uno, entonces solo pueden ser $1$ o $-1$. Por lo tanto $\mathbf M$ y $\mathbf M'$ son unimodulares
 }
 \alt<2>{\qedhere}{\phantom\qedhere}
\end{proof}
    
\end{frame}

\section{Complejidad}

\section{Algunos problemas de reticulados en la criptografía}

\begin{frame}
    \begin{center}
        {\huge \bf Preguntas}
    \end{center}
\end{frame}

\end{document}

