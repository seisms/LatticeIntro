\documentclass{beamer}

\usepackage{subcaption, float}
\usepackage{graphicx} % Required for inserting images
\usepackage{svg}
\usepackage[spanish]{babel}

%%%%%%%%%% Font %%%%%%%%%%%
\usepackage[T1]{fontenc}
\usepackage{amssymb, amsfonts}
\usepackage{amsbsy}
%%%%%%%%%%%%%%%%%%%%%%%%%%%


%%%%%%%%%% Math %%%%%%%%%%%
% Useful tools like equation environments, alignment, etc.
\usepackage{amsthm, amsmath}

\theoremstyle{remark}
\newtheorem*{remark}{Observación}

% Para cambiar a los símbolos "tradicionales" de
% de los naturales reales etc. se tiene que cambiar
% \mathbf por \mathbb. Me gusta más dejarlos en negrita
% porque encuentro que es más legible en presentaciones,
% pero si quieren lo cambian.
\newcommand{\ZZ}{\mathbb{Z}}
\newcommand{\QQ}{\mathbb{Q}}
\newcommand{\RR}{\mathbb{R}}
\newcommand{\MM}{\mathbb{M}}
\newcommand{\NN}{\mathbb{N}}
\newcommand{\latt}{\mathcal{L}(\mathbf{B})}       % Para escribir menos
\newcommand{\lattprime}{\mathcal{L}(\mathbf{B'})} % Para escribir menos

%%%%%%%%%%%%%%%%%%%%%%%%%%%%


%%%% Beamer specific %%%%%%
\title{Introducción a Reticulados}
% Pongan sus apellidos maternos en ...
\author{Leandro Aballay Henríquez, Diego Cuevas Alarcón, Matías Olivares Morales, Álvaro Quezada Inostroza, Guillermo Pereira Bula}
\institute{Departamento de Matemática y Ciencia de la Computación\\}
\date{17 de julio de 2025}

\AtBeginSection[]
{
  \begin{frame}
    \frametitle{Contenidos}
    \tableofcontents[currentsection]
  \end{frame}
}

\uselanguage{Spanish}
\languagepath{Spanish}
\newcommand{\adv}[1]{ {\color{blue} #1} }


\usetheme{default}
\usecolortheme{default}
%%%%%%%%%%%%%%%%%%%%%%%%%%%
